\documentclass{paper}
\usepackage[utf8]{inputenc}
\usepackage{float}
\usepackage{graphicx}
\graphicspath{ {images/} }
\title{Un algoritmo de dos fases para reconocer actividades humanas en el contexto de la Industria 4.0 y los procesos antropogenicos.}
\author{Borja Bordel\textsuperscript{1}, Ramon Alcarria1\textsuperscript{1}, Diego Sanchez-de-Rivera1\textsuperscript{1}}



\pagestyle{empty}
\begin{center}
\maketitle
\textsuperscript{1}Universidad Politecnica de Madrid
\\Madrid, España
\\bbordel@dit.upm.es, ramon.alcarria@upm.es, diegosanchez@dit.upm.es
\end{center}
\bigskip

\begin{document}

\textbf{Resumen} Sistemas industriales futuros, una revolucion conocida como Industria 4.0 son concebidos para integrar a las personas en un mundo cibernetico como prosumidores (proveedores de servicio y consumidores). En este contexto, los procesos antropogenicos aparecen como una realidad esencial y un instrumento para crear bucles de informacion de retroalimentacion entre el subsistema social (gente) y el subsistema cibernetico (componentes tecnologicos) son requeridos. Aunque muchos instrumentos distintos han sido propuestos, a dia de hoy las tecnicas de reconocimiento de patrones  son las mas prometedoras. Por ejemplo, son dependientes del hardware seleccionado para adquirir informacion de los usuarios; o presentan un limite de precision del proceso de reconocimiento. Para abordar esta situacion, en este articulo se propone un algoritmo de dos fases para integrar a la gente en los sistemas de la Industria 4.0 y los procesos antropogenicos. El algoritmo define acciones complejas como composicion de movimientos simples. Las acciones complejas son reconocidas usando Modelos Ocultos de Markov, y los movimientos simples son reconocidos usando la Deformacion Dinamica del Tiempo. De esa forma, solo los movimientos dependen del hardware para captar la informacion, y la precision del reconocimiento de acciones complejas se vuelve enormemente mejorada. Una validacion experimental real es tambien llevada a cabo para evaluar y comparar la presentacion de la solucion propuesta.

\bigskip
\textbf{Palabras clave:}Industria 4.0; reconocimiento de patrones, la Deformacion Dinamica del Tiempo; Inteligencia Artificial; Modelos Ocultos de Markov



\section{Introduccion}

Industria 4.0 [1] hace referencia al uso de Sistemas Ciberfisicos (union de procesos fisicos y ciberneticos) [2] como el componente tecnologico principal en soluciones digitales futuras, principalmente (pero no unicamente) en escenarios industriales. Normalmente, la digitalizacion ha causado, al final, que los trabajadores en las lineas de ensamble fueran sustituidos por robots durante la tercera revolucion industrial.
Sin embargo, algunas aplicaciones industriales no pueden basarse en soluciones tecnologicas, siendo el trabajo humano aun esencial [3]. Los productos hechos a mano son un ejemplo de las aplicaciones donde la presencia del trabajo humano es esencial. Estos sectores industriales, en cualquier caso, debe ser tambien integrado en la cuarta revolucion industrial. De la union de los Sistemas Ciberfisicos (CPS) y los humanos actuando como proveedores de servicio (trabajos activos), surgimiento del CPS humanizado [4]. En estos nuevos sistemas, los procesos antropogenicos son permitidos [5]; i.e. los procesos que son conocidos, ejecutados y manejados por personas (a pesar de que deben ser vigilados por mecanismos digitales).
Para crear una integracion real entre personas y tecnologia, y mover la ejecucion del proceso del subsistema social (humanos) al mundo cibernetico (componentes de hardware y software), las tecnicas de la extraccion de la informacion son necesarias. Muchas soluciones diferentes y acercamientos han sido reportados durante los ultimos años, pero las tecnicas de reconocimiento de patrones son las mas prometedoras.
El uso de Inteligencia Artificial, modelos estadisticos y otros instrumentos similares han permitido un real e increible desarrollo de soluciones de reconocimiento de patrones, pero algunos desafios todavia siguen pendientes.
Primero, las tecnicas de reconocimiento de patrones dependen del hardware subyacente para captar informacion. La estructura y los cambios en el proceso de aprendizaje si (por ejemplo) en vez de acelerometros que  consideramos sensores infrarrojos. Esto es muy problematico debido a que las tecnologias de hardware evolucionan mucho mas rapido que las soluciones de software.
Y, segundo, hay un limite en la precision en el proceso de reconocimiento. En realidad, como las acciones humanas se vuelven mas complicadas, mas variables y mas modelos complejos son requeridos para reconocerlos. Esta aproximacion genera grandes problemas de optimizacion cuyo error residual es mas alto segun aumenta el numero de variables; lo que lo que provoca una disminucion en la tasa de exito del reconocimiento [6]. En conclusion, las matematicas (no el software, por lo que, no depende de la implementacion) fuerza una cierta precision para el proceso del reconocimiento de patrones haciendo que las acciones sean estudiadas. Para prevenir esta situacion, se debe considerar un numero mas bajo de variables, pero esto tambien reduce la complejidad de las acciones que deben ser analizadas; una solucion que no es aceptable en escenarios industriales donde las actividades de produccion complejas son desarrolladas.
Por lo tanto, el objetivo de este articulo es describir un nuevo algoritmo de reconocimiento de patrones abordando estos dos problemas basicos. El mecanismo propuesto define las acciones como una composicion de movimientos simples. Los movimientos simples son reconocidos usando las tecnicas de la Deformacion Dinamica del Tiempo (DTW) [7]. Este proceso es dependiente de un hardware concreto para captar la informacion; pero la DTW es muy flexible y actualizar el repositorio de patrones es suficiente para reconfigurar todo el algoritmo. Luego las acciones complejas son reconocidas como combinaciones de movimientos simples a traves de los Modelos Ocultos de Markov (HMM) [8]. Estos modelos son totalmente independientes de las tecnologias de hardware, ya que solo se consideran como acciones simples. Este acercamiento de dos fases tambien reduce la complejidad de los modelos, incrementando la precision y la tasa de exito en el proceso de reconocimiento.
El resto del articulo esta organizado de esta manera: La seccion 2 describe el estado del arte en el reconocimiento de patrones para las actividades humanas; la seccion 3 describe la solucion propuesta, incluyendo las dos fases definidas; la seccion 4 presenta un validacion experimental usando un escenario real y usuarios finales; y la seccion 5 cierra el articulo.


\section*{2  Estado del arte del reconocimiento de un patron}
\ Varias tecnicas de reconocimiento de patrones para actividades humanas han sido desarrolladas. Sin embargo la propuesta mas usual puede clasificarse en cinco categorias basicas [9]: (i) Cadenas de Markov; (ii) La Cadena Aleatoria 
Condicional de Saltos; (iii) Patrones Emergrntes; (iv) El Campo Aleatorio Condicional; (v) Clasificadores Bayesianos.

\ De hecho, la mayoria de los autores proponen el uso de Cadenas de Markov(HMM) para modelar actividades humanas.
      HMM permite modelar acciones como cadenas de Markov [10][11]. Basicamente,
HMM genera estados ocultos a partir de datos observables.
En particular, el objetivo final de
esta tecnica es para construir la secuencia de estados ocultos que se ajusta a una cierta secuencia de datos.Para finalmente definir todo el modelo, HMM debe deducir a partir de los datos los modelos parametricos de manera confiable. La figura 1  muestra una representacion esquematica del funcionamiento del HMM. Cuando se reconocen las actividades humanas, las acciones que componen las actividades son los estados ocultos y las salidas de los sensores son los datos que se estudian. HMM, ademas, permitir el uso de tecnicas de entrenamiento considerando conocimientos previos sobre el modelo. Este entrenamiento a veces es esencial para “inducir” todas las posibles secuencias de datos requeridas para calcular el HMM. Finalmente, es muy importante tener en cuenta que los HMM aislados simples se pueden combinar para crear modelos mas grandes y complejos.


\ \begin{figure}[H]
    \centering
    \includegraphics{latex2.png}
    \caption{Representacion grafica de un HMM}
    \label{fig:my_label}
\end{figure}

\ Los HMM, sin embargo, son inutiles para modelar ciertas actividades concurrentes, por lo que otros autores han reportado una nueva tecnica denominada Conditional Random Field (CRF).
CRF se emplean para modelar aquellas actividades que presentan acciones concurrentes o, en
general, multiples acciones que interactuan [12][13]. Ademas, HMM requiere un gran esfuerzo de entrenamiento para descubrir todos los estados ocultos posibles. Para resolver estos problemas, el campo aleatorio condicional (CRF) emplea probabilidades condicionales en lugar de distribuciones de probabilidad conjunta. Asi, las actividades cuyas acciones se desarrollen en cualquier orden se puede modelar facilmente. A diferencia de las cadenas en HMM, CRF emplea graficos aciclicos y permite la integracion de estados ocultos condicionales (estados que dependen de observaciones pasadas y/o futuras). 




\ Los CRF, por otro lado, siguen siendo inutiles para modelar ciertos comportamientos, por lo que algunas propuestas generalizan este concepto y proponen el Skip Chain Conditional Random Field (SCCRF). SCCRF es una tecnica de reconocimiento de patrones, mas general que CRF, que permite modelar actividades que no son una secuencia de acciones en la naturaleza [14]. Esta tecnica trata de capturar dependencias de largo alcance (cadena de salto); y puede entenderse como el producto de diferentes cadenas lineales. Sin embargo, calcular este producto es bastante pesado y complicado, por lo que esta tecnica suele ser demasiado costosa desde el punto de vista computacional para implementarla en pequeños sistemas integrados. 

\ Otras propuestas emplean tecnicas de descripcion de mayor nivel como Emerging Patterns (EP). Para la mayoria de los autores, EP es una tecnica que describe actividades como vectores de parametros y sus valores correspondientes (ubicacion, objeto, etc.) [15]. Usando
distancias entre vectores es posible calcular y reconocer acciones desarrolladas por personas. Finalmente, otros autores han empleado con exito tecnicas secundarias como los clasificadores bayesianos [16], que identifican actividades haciendo una correspondencia entre las actividades humanas y las salidas de sensores mas probables mientras estas acciones
se realizan considerando que todos los sensores son independientes. Los arboles de decision [17], las extensiones HMM [18] y otras tecnologias similares tambien se han estudiado en la literatura, aunque estas propuestas son escasas. 

\ Entre todas las tecnologias descritas, HMM no es la mas poderosa. Sin embargo, encaja a la perfeccion con la Industria 4.0, donde las actuaciones son muy complejas pero muy estructuradas y ordenadas (segun protocolos de empresa, politicas de eficiencia, etc.). Ademas, se requiere una retroalimentacion rapida (a veces incluso en tiempo real) para garantizar que los procesos impulsados por humanos funcionen correctamente antes de que ocurra una falla critica global. Por lo tanto, las soluciones computacionalmente costosas no son un enfoque valido, y estamos seleccionando HMM como tecnologia de base principal. Para preservar su caracter liviano y, al mismo tiempo, poder modelar actividades complejas, introducimos un esquema de reconocimiento de dos fases que permite dividir acciones complejas en dos pasos mas simples.


\section*{3  Algoritmo de reconocimiento de patrones de dos fases}

Con el fin de (i) independizar el proceso de reconocimiento de patrones de los dispositivos de hardware empleados para capturar informacion, (ii) permitir el reconocimiento de acciones complejas y (iii) preservar el caracter liviano de los modelos seleccionados, la solucion propuesta presenta una arquitectura con tres capas diferentes (ver Figura 2).


\ \begin{figure}[H]
    \centering
    \includegraphics{latex4.png}
    \caption{Arquitectura de la solucion de reconocimiento de patrones propuesta}
    \label{fig:my_label}
\end{figure}


\ La capa mas baja incluye la plataforma de hardware. Los dispositivos de monitoreo como acelerometros, telefonos inteligentes, sensores infrarrojos, etiquetas RFID, etc., se implementan para capturar informacion sobre el comportamiento de las personas. Las salidas de estos dispositivos crean
secuencias de datos fisicos cuyo formato, rango dinamico, etc., dependen totalmente de las tecnologias de hardware seleccionadas.


\ Estas secuencias de datos fisicos luego se procesan en la capa intermedia utilizando tecnicas DTW. Como resultado, para cada secuencia de datos fisicos, un simple movimiento o accion es reconocido. Estas acciones simples se representan mediante un formato de datos binarios para que la solucion sea lo mas ligera posible. El software de este nivel debe modificarse cada
vez que se actualiza la plataforma de hardware, pero las tecnologias DTW no requieren un proceso de actualizacion pesado, y actualizar el repositorio de patrones es suficiente para configurar el algoritmo en este nivel.

\ Los movimientos simples reconocidos, entonces, se agrupan para crear secuencias de datos logicos. Estas secuencias alimentan un sistema de reconocimiento de patrones de alto nivel basado en mCademas de Markov. En este nivel, los componentes de software requieren un proceso de entrenamiento pesado, pero la capa intermedia hace que la plataforma de hardware y los modelos de alto nivel sean totalmente independientes. Por lo tanto, cualquier cambio en la plataforma de hardware no impone una actualizacion en el HMM, lo que seria extremadamente costoso computacionalmente. Mediante el analisis de la secuencia de movimientos simples, se reconocen acciones complejas. La siguiente subseccion describe en detalle las dos fases de reconocimiento de patrones propuestas.



\subsection*{ 3.1 Reconocimiento de movimiento simple: deformacion dinamica del tiempo}

\lipsum[Con el objetivo de reconocer gestos o movimientos simples, se ha seleccionado una Deformacion Dinamica del Tiempo. Las tecnologias de Deformacion Dinamica del Tiempo contienen los requerimientos de un software de medio nivel dado que se adaptan a las caracteristicas de la plataforma del hardware subyacente muy facilmente y son relativamente rapidas y eficientes (algunos diminutos dispositvos incrustados pueden contenerlas).] 


\ En nuestra solucion, el factor humano esta monitorado mediante una serie de sensores S, conteniendo $N_S$  componentes.

\ S = { $s_i$, i = 1, … , $N_S$ }

\ La produccion de estos sensores son periodicamente mostradas cada $T_s$ segundos; obteniendo para cada instante de tiempo, t, un vector de $N_s$ valores (cada valor de cada sensor). Este vector $Y_t$ es llamado una "muestra multidimensional".

\ $Y_t = {y_t^i , i = 1, ..., N_S}$

\ Entonces, un simple movimento Y tendra una duracion de $T_m$ segundos y sera descrito por la secuencia temporal de $N_m$ muestras multidimensionales que habran sido coleccionadas en ese tiempo. Para reconocer movimientos mas tarde, se ha creado un almacen $\mathcal{R}$ conteniendo las correspondientes acciones temporales para que cada una de las acciones simples K sean reconocidas.

\ Y = {$Y_t$, t = 1, ..., T_m} = {$Y^i$, i = 1, ..., $N_m$}

\ $\mathcal{R}$ = {$R_i$, i = 1, ..., K}

\ En general, la gente realiza movimientos similares pero de diferentes maneras. Ademas, las transiciones pueden ser mas lentas o mas rapidas, algunas acciones elementales pueden ser añadidas o eliminadas, etc... Por ello, dada una secuencia X con $N_x$ muestras, representando a un movimiento a ser reconocido, debe estar localizado el patron $R_i$ $\in$ $\mathcal{R}$ mas cercano a X; de la forma que $R_i$ sea reconocida como la accion realizada. Para hacer eso es definida una funcion de distancia. Esta funcion de distancia puede ser aplicada para calcular una matriz de coste, requiriendo tantas muestras frecuentemente no tienen la misma longitud ni estan alineadas.

\ d: $\mathcal{F}$ X $\mathcal{F}$  \longrightarrow $\mathfrak{R}^+$   ,    $X_i$, $R_i^j$ $\in$ $\mathcal{F}$

\ C $\in$ $\mathfrak{R}^{N_x x N_m}$  C(n,m) = d($X^n$, $R^m_j$)

\ En los sensores posicionales (acelerometros, dispositivos infrarrojos, etc.) la funcion distancia es aplicada directamente a la produccion de estos (al contrario que, por ejemplo, microfonos, cuya produccion debe ser evaluada en el dominio de potencia). Aunque pueden ser empleadas otras funciones de distancia (como la divergencia de Kullback-Leibler o la distancia de Manhattan), para este primer trabajo estamos empleando la distancia estandar Euclidea.

\ d($X^n$, $R_j^m$) = $\sqrt{
\sum(x_i^n - r_i^{m,j})^2}$

\ Despues, es definido un camino torcido  p = ($p_1$,$p_2$,...,$p_L$) como secuencia de pares ($n_\mathfrak{l}, m_\mathfrak{l})$ $\in$ [1, $N_x$] X [1, $N_m$] and $\mathfrak{l}$ $\in$ [1, L], satisfaciendo tres condiciones: (i) la condicion vinculante, i.e., $p_1$ = [1,1] y $p_L$ = [$N_x$,$N_m$]; (ii) la condicion monotonica, i.e. $n_1 \leq n_2  \leq n_3 ...  \leq n_L$; y (iii) la condicion de escalon de tamaño, i.e. $p_\mathfrak{l}$ -$p_{\mathfrak{l}-1}$ $\in$ {(1,0), (0,1), (1,1)} with \mathfrak{l} $\in$ [1, L-1].

\ Entonces, el coste total de un camino torcido $p_i$ es calculado añadiendo todos los costes parciales o distancias. Con todo esto, la distancia entre dos secuencias de datos $R_i$ y X es definida como el coste (distancia) del camino torcido optimo $p^*$.

\ $d_{p_i}$(X,$R_i$) = \sum($X^{n_l}$, $R_j^{m_l}$)

\ $d_{PTW}$(X,$R_j$) = $d_p^*($X,$R_j$) = min{$d_{p_i}$(X,$R_j$), siendo $p_i$ un camino torcido}

\ Finalmente, el simple movimiento reconocido desde la secuencia de datos X es aquel cuyo patron $R_i$ tiene la menor distancia (es el mas cercano) a X. El uso de esta definicion es tolerante a variaciones de velocidad en la ejecucion del movimiento, a la introduccion de micro-gestos, etc... Ademas, como se puede ver, cuando una tecnologia de hardware diferente es empleada, es suficiente con actualizar el patron de repositorio $\mathcal{R}$ para reconfigurar el patron de reconocimiento de solucion entero (dado que no es requerido ningun entrenamiento).


\subsection*{3.2 Reconocimiento de accion compleja: Cadenas Ocultas de Markov}

\ El mecanismo propuesto previamente es muy util para reconocer acciones simples, pero las acciones mas complejas involucran un numero mas alto de variables y requieren de mucho mas tiempo. Ademas, la Deformacion Dinamica del Tiempo tiende a ser mas imprecisa, y se requieren de modelos probabilisticos. Entre todos los modelos existentes, las Cadenas Ocultas de Markov son el mas adecuado para escenarios industriales y procesos humanos.

\ Desde la fase anterior, el universo de los posibles movimientos simples a ser reconocidos es \mathcal{M} = {$m_i$, i = 1, ..., K}. Aparte, se define un estado universal \mathcal{U} = {$u_i$, i = 1, ..., Q}, describiendo todos los estados que la gente puede cruzar al realizar cualquiera de las acciones bajo estudio.

\ Despues, tambien es considerado un set de observaciones \mathcal{O} = {$O_i$, i = 1, ..., $Z_O$} (movimientos simples reconocidos en la fase anterior), al igual que la secuencia de estados V = {$O_i$, i = 1, $Z_v$} describiendo la accion a ser modelada por la Cadena Oculta de Markov. En este caso inicial, estamos asumiendo que cada observacion corresponde a un nuevo estado, asi que $Z_v$ = $Z_o$. Entonces, tres matrices son calculadas: (i) la matriz transitoria A, decribiendo la probabilidad de que el estado $u_j$ siga al estado $u_i$; (ii) la matriz de observacion B, describiendo la probabilidad de que la observacion $o_i$ sea causada por el estado $u_j$ independientemente de k; y (iii) la matriz de probabilidad inicial \prod.

\ A = [$a_{i,j}$]   $a_{i,j}$ = P($v_k$ = $u_j$ | $v_{k-1}$ = $u_j$)

\ B = [$b_j$($o_i$)]  $b_j$($o_i$) = P($x_k$ = $o_i$ | $v_k$ = $u_j$)

\ \prod = [$\pi_i$]   $\pi_i$ = P($v_i$ = $u_i$)

\ Entonces, cada Cadena Oculta de Markov para cada actividad compleja $\lambda_i$ a reconocer es descrita mediante estos tres previos elementos.

\ $\lambda_i$ = [A, B, \prod]

\ Ademas, se realizan dos asumciones: (i) la asumcion de Markov, mostrando que todo estado es solamente dependiente del anterior; y (ii) la asumcion de independencia, declarando que cualquier secuencia de observaciones depende solo del estado presente y no de estados u observaciones previos.

\ P($v_k$ | $v_1$, ..., $v_(k-1)$) = P($v_k$ | $v_(k-1)$)

\ P($o_k$ | $o_1$, ..., $o_(k-1)$, $v_1$, $v_k$) = P($o_k$ | $v_k$)

\ Para evaluar el modelo y reconocer la actividad que esta siendo realizada por los usuarios, en este paper estamos usando un acercamiento tradicional. Aunque algoritmos avanzados han sido provados ser mas eficientes, para este trabajo inicial estamos implementando directamente la expresion de evaluacion en su forma tradicional.

\ P($\theta$|$\lambda$) = \sum P($\theta$|V,$\lambda$) P(V|$\lambda$) =

\ = \sum( \prod P($o_i$|$v_i$,$\lambda$) )($\pi_{v1}$ · $a_{v1v2}$ · ... · $a_{v_{zv-1}v_{zv}}$) =

\ = \sum $\pi_{v1}$ · $b_{v1}(o_1)$ · $a_{v1v2}$ · $b_{v2}(o_2)$ · ... · $a_{v_{zv-1}v_{zv}}$ · $b_{vzv}(o_{zo})$

\ El proceso de aprendizaje tambien fue implementado de la manera mas simple. Para la matriz transitoria, la de observacion y la inicial se emplearon definiciones estadisticas. En particular, la definicion de probabilidad de Laplace fue empleada para estimar estas tres matrices desde estadisticos acerca de las actividades bajo estudio. El operador count(·) el numero de veces que un evento ocurre.

\ $a{i,j}$ = P($u_j$ | $u_i$) = $\frac{count($u_j$ sigue $u_i$}{count($u_j$)}$

\ $b_j$($o_i$) = P($o_i$|$u_j$) = $\frac{count($o_i$ es observado en el estado $u_j$}{count($u_j$)}$

\ $\pi_{i}$ = P($v_1$ = $u_1$) = $\frac{count($v_1$ = $u_1$}{count($v_1$)}$




\subsection*{4 Validacion experimental: implementacion y resultados}

Con el fin de evaluar el desempeño de la solucion propuesta, se diseño una validacion experimental y se llevo a cabo. Un escenario industrial fue simulado en aulas grandes de la Universidad Politecnica de Madrid. El escenario representaba una empresa tradicional de productos artesanales hechos a mano. En particular, un pequeño fabricante de PCI (Placas de Circuito Impreso) fue emulado.
Para captar informacion sobre el comportamiento de las personas, se proporciono a los participantes con un guante cibernetico, que incluia acelerometros y un lector RFID [19]. Los objetos alrededor de los escenarios se identificaron con una etiqueta RFID, por lo que la plataforma de hardware propuesta pueda identificar la posicion de la mano (gesto) y los objetos con los que interactuan las personas.
Se definio y reconocio una lista de doce actividades complejas diferentes utilizando la tecnologia propuesta. La tabla 1 describe las doce actividades definidas, incluyendo una una breve descripcion de las mismas. 

\ \begin{figure}[H]
    \centering
    \includegraphics{tabla3.png}
    \label{fig:my_label}
\end{figure}


Dieciocho personas (18) participaron en el experimento. Se pidio a las personas que realizaran las actividades en un orden aleatorio. El orden real, asi como el orden en que las actividades se reconocen fueron almacenados por un proceso de software de supervision. La tasa de exito para toda la solucion se evaluo, identificando (ademas), la mismatasa para cada una de las fases existentes.
Para evaluar la mejora obtenida en comparacion con las soluciones similares existentes, se emplearon las mismas secuencias de datos fisicos para alimentar una solucion de reconocimiento de patrones estandar basado unicamente en HMM. Utilizando un software de procesamiento estadistico de datos se extrayeron algunos resultados relevantes.
La figura 4 representa la tasa de exito media para tres casos: la solucion global, la primera fase (DTW) y la segunda fase (HMM). Ademas, tambien se incluye el porcentaje de exito del enfoque tradicional basado en HMM. Como puede verse, la tecnologia propuesta es, globalmente, alrededor de un 9 por ciento mejor que las tecnicas tradicionales de reconocimiento de patrones
tradicionales basadas en HMM exclusivamente. Ademas, la primera fase (basada en DTW) es un 20 por ciento peor que la segunda fase (HMM), lo cual es significativo ya que las tecnicas de las tecnicas de Time Warping son mas debiles por defecto. 

\ \begin{figure}[H]
    \centering
    \includegraphics{succes.png}
    \caption{Fig. 4. Tasa media de exito de la solucion propuesta}
    \label{fig:my_label}
\end{figure}



\subsection*{ 5  Conclusiones y futuros trabajos}

En este trabajo presentamos un nuevo algoritmo de reconocimiento de patrones para integrar a las personas en los sistemas de la Industria 4.0 y los procesos impulsados por el hombre. El algoritmo define actividades complejas
como composiciones de movimientos simples. Las actividades complejas se reconocen mediante modelos de Markov ocultos, y los movimientos simples se reconocen mediante Time Warping. Para permitir la aplicacion de este algoritmo en pequeños dispositivos embebidos, se seleccionan configuraciones ligeras. Tambien se lleva a cabo una validacion experimental, y los resultados muestran una mejora global de la tasa de exito en torno al 9.
En futuros trabajos se consideraran las metodologias mas complejas para el procesamiento de datos, y se evaluara la comparacion para diferentes configuraciones del algoritmo propuesto. Ademas, se analizara la propuesta en diferentes escenarios.

AGRADECIMIENTOS. La investigacion que ha conducido a estos resultados ha recibido financiacion del el Ministerio de Economia y Competitividad a traves del proyecto SEMOLA (TEC2015-68284-R) y del Ministerio de Ciencia, Innovacion y Universidades a traves de VACADENA (proyecto RTC-2017-6031-2).


\subsection*{Referencias}

\bigskip
\bigskip

1. Bordel, B., Alcarria, R., Sanchez-de-Rivera, D., & Robles, T. (2017, noviembre). Proteger los sistemas de la industria 4.0 contra los efectos maliciosos de los ataques ciberneticos. En
Conferencia Internacional sobre Computacion Ubicua e Inteligencia Ambiental (pags. 161-171). Springer, Cham.

\bigskip
\bigskip

 2. Bordel, B., Alcarria, R., Robles, T., & Martin, D. (2017). Sistemas ciberfisicos:
Ampliacion de la deteccion generalizada de la teoria del control al Internet de las cosas. Informatica omnipresente y movil, 40, 156-184.

\bigskip
\bigskip

3. Neff, W. (2017). Trabajo y conducta humana. Routledge.

\bigskip
\bigskip

4. Bordel, B., Alcarria, R., Martin, D., Robles, T., & de Rivera, D. S. (2017). Autoconfiguracion en sistemas ciberfisicos humanizados. Revista de Inteligencia Ambiental y Computacion Humanizada, 8(4), 485-496.

\bigskip
\bigskip

5. Bordel, B., de Rivera, D. S., Sanchez-Picot, A., & Robles, T. (2016). Control de procesos fisicos en sistemas basados en la industria 4.0: un enfoque en los sistemas ciberfisicos. en omnipresente
Computacion e Inteligencia Ambiental (pags. 257-262). Springer, Cham.

\bigskip
\bigskip

6. Pal, S. K. y Wang, P. P. (2017). Algoritmos geneticos para el reconocimiento de patrones. Prensa CRC.

\bigskip
\bigskip


7. Muller, M. (2007). Deformacion dinamica del tiempo. Recuperacion de informacion para musica y movimiento, 69-84.

\bigskip
\bigskip

8. Eddy, S. R. (1996). Modelos ocultos de markov. Opinion actual en biologia estructural, 6(3), 361-365.

\bigskip
\bigskip


9. Kim, E., Helal, S. y Cook, D. (2010). Reconocimiento de la actividad humana y descubrimiento de patrones. IEEE Pervasive Computing/IEEE Computer Society [y] IEEE Communications
Sociedad, 9(1), 48.

\bigskip
\bigskip

10. Li, Z., Wei, Z., Yue, Y., Wang, H., Jia, W., Burke, L. E., ... & Sun, M. (2015). An adaptive hidden markov model for activity recognition based on a wearable multi-sensor device. Journal of medical systems, 39(5), 57.

\bigskip
\bigskip

11. Ordonez, F. J., Englebienne, G., De Toledo, P., Van Kasteren, T., Sanchis, A., & Krose, B. (2014). In-home activity recognition: Bayesian inference for hidden Markov models. IEEE Pervasive Computing, 13(3), 67-75.

\bigskip
\bigskip

12. Zhan, K., Faux, S., & Ramos, F. (2015). Multi-scale conditional random fields for firstperson activity recognition on elders and disabled patients. Pervasive and Mobile Computing, 16, 251-267.

\bigskip
\bigskip


13. Liu, A. A., Nie, W. Z., Su, Y. T., Ma, L., Hao, T., & Yang, Z. X. (2015). Coupled hidden conditional random fields for RGB-D human action recognition. Signal Processing, 112, 74-82.

\bigskip
\bigskip


14. Liu, J., Huang, M., & Zhu, X. (2010, July). Recognizing biomedical named entities using skip-chain conditional random fields. In Proceedings of the 2010 Workshop on
Biomedical Natural Language Processing (pp. 10-18). Association for Computational Linguistics

\bigskip
\bigskip



15. Gu, T., Wu, Z., Tao, X., Pung, H. K. y Lu, J. (marzo de 2009). epsicar: un enfoque basado en patrones emergentes para el reconocimiento de actividades secuenciales, intercaladas y concurrentes. En
Computacion y comunicaciones generalizadas, 2009. PerCom 2009. Conferencia internacional IEEE sobre (pags. 1-9). IEEE.

\bigskip
\bigskip


16. Hu, B. G. (2014). ¿Cuales son las diferencias entre los clasificadores bayesianos y los clasificadores de informacion mutua?. Trans. IEEE. Red neuronal Sistema de aprendizaje, 25(2), 249-264.

\bigskip
\bigskip

17. Wang, X., Liu, X., Pedrycz, W. y Zhang, L. (2015). Arboles de decision basados en reglas difusas. Reconocimiento de patrones, 48(1), 50-59.

\bigskip
\bigskip

18. Davis, M. H. (2018). Modelos de Markov y optimizacion. Routledge.

\bigskip
\bigskip

19. Bordel Sanchez, B., Alcarria, R., Martin, D., & Robles, T. (2015). TF4SM: un marco para desarrollar soluciones de trazabilidad en pequeñas empresas manufactureras. Sensores, 15(11), 29478-29510.


\end{document}





